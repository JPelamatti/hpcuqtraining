\usepackage{graphicx}
\unitlength=1cm
\graphicspath{{./figures/}{../../../Scripts/Metamodel/}}

\usepackage[space]{grffile}

\usepackage{alltt}
\usepackage{euscript}
\usepackage{hyperref}
\usepackage{bbm}

\usepackage{multicol}

\usepackage[utf8]{inputenc}
\usepackage[T1]{fontenc}

\usepackage{listings}
\definecolor{darkgreen}{rgb}{0,0.5,0}
\definecolor{violet}{rgb}{0.5,0,1}

% Python macros
\newcommand{\pyvar}[1]{\texttt{#1}}

% https://tex.stackexchange.com/questions/183149/cant-silence-a-pdftex-pdf-inclusion-multiple-pdfs-with-page-group-error
\pdfsuppresswarningpagegroup=1

% Configuration beamer
\usetheme{Montpellier}
\setbeamertemplate{navigation symbols}{} % Remove navigation
\useoutertheme{infolines}
\setbeamertemplate{theorems}[numbered] 
% Utilise des fonts serif, pour éviter les pb de fonte
\usefonttheme{serif} 
\setbeamertemplate{caption}[numbered]


% Commandes mathématiques
\newcommand{\RR}{\mathbb{R}}
\newcommand{\ZZ}{\mathbb{Z}}
\newcommand{\esp}{\mathbb{E}}
\newcommand{\var}{\mathbb{V}}
\newcommand{\cov}{\textbf{Cov}}
\newcommand{\likelihood}{\mathcal{L}} % Likelihood

% Lettres en gras
\newcommand{\bc}{{\bf c}}
\newcommand{\bT}{{\bf T}}
\newcommand{\bX}{{\bf X}}
\newcommand{\bY}{{\bf Y}}
\newcommand{\bZ}{{\bf Z}}
\newcommand{\bg}{{\bf g}}
\newcommand{\bs}{{\bf s}}
\newcommand{\bx}{{\bf x}}
\newcommand{\by}{{\bf y}}
\newcommand{\bz}{{\bf z}}
\def\bzero{{\bf 0}}

\newcommand{\bindicator}{\mathbbm{1}}
\newcommand{\red}[1]{\textcolor{red}{#1}}

% Symboles mathématiques gras
\def\balpha{\boldsymbol{\alpha}}
\def\bbeta{\boldsymbol{\beta}}
\def\btheta{\boldsymbol{\theta}}
\def\bmu{\boldsymbol{\mu}}
\def\bxi{\boldsymbol{\xi}}
\def\bpsi{\boldsymbol{\psi}}

\renewcommand{\P}{\mathbb{P}}
% From https://github.com/openturns/openturns/blob/master/python/doc/math_notations.sty
\DeclareMathOperator*{\argmin}{argmin}
\DeclareMathOperator*{\argmax}{argmax}

% Binomial coefficient
\newcommand{\choosefun}[2]{
\left(
\begin{array}{@{}c@{}}
#1\\
#2
\end{array}
\right)
}

\newtheorem{counterexample}{Couter-example}
\newtheorem{hypothesis}{Hypothesis}
\newtheorem{hypotheses}{Hypotheses}

